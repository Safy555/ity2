\documentclass[a4paper, 11pt, twocolumn]{article}

\usepackage[utf8]{inputenc}
\usepackage[czech,shorthands=off]{babel}
\usepackage[T1]{fontenc}
\usepackage{lmodern}
\usepackage[left=1.3cm, text={18.6cm, 26cm}, centering, top=1.8cm]{geometry}
\usepackage{hyperref}
\usepackage{amsthm, amssymb, amsmath}

\theoremstyle{definition}
\newtheorem{definition}{Definice}

\theoremstyle{definition}
\newtheorem{sentence}{Věta}

\begin{document}

\thispagestyle{empty}


\begin{titlepage}
\begin{center}
    
    \Huge{
    \textsc{
    Vysoké učení technické v Brně\\[0.5em]}
    }
    \huge{
    \textsc{
    Fakulta informačních technologií}\\
    \vspace{\stretch{0.382}}
    }
    {\LARGE
    Typografie a publikování -- 2. projekt\\[0.6em]
    Sazba dokumentů a matematických výrazů\\
    }
    \vspace{\stretch{0.618}}
\end{center}

{\Large 2025 \hfill
Jakub Králik (xkralij00)}

\end{titlepage}


\section*{Úvod}
V této úloze vysázíme titulní stranu a ukázku matematického textu,
v němž se vyskytují například
rovnice \eqref{eq7} na straně \pageref{def2}, Věta \ref{veta1} nebo Definice \ref{def2}.
Pro vytvoření těchto odkazů používáme kombinace příkazů
\texttt{\textbackslash label}, \texttt{\textbackslash ref}, \texttt{\textbackslash eqref} a \texttt{\textbackslash pageref}.
Před odkazy patří nezlomitelná mezera.
Text zvýrazníme pomocí příkazu \texttt{\textbackslash emph}, strojopisné písmo pomocí \texttt{\textbackslash tesxttt}.
Pro \LaTeX ové příkazy (s obráceným lomítkem) použijeme \texttt{\textbackslash verb}.

Titulní strana je vysázena prostředím \texttt{titlepage} a~ nadpis je v optickém středu
s využitím \emph{zlatého řezu}, který byl probrán na přednášce.
Na titulní straně jsou tři různé velikosti písma a mezi dvojicemi řádků textu
je řádkování se zadanou  velikostí 0,5\,em a 0,6\,em\footnote{Použijte správnou velikost mezery mezi číslem a jednotkou.}.

\section{Matematický text}
Symboly číselných množin sázíme makrem \texttt{\textbackslash mathbb},
kaligrafická písmena  makrem \texttt{\textbackslash mathcal}.
Pozor na tvar i sklon řeckých písmen: srovnejte \texttt{\textbackslash rho} a \texttt{\textbackslash varrho}.
Konstrukce \texttt{\$\{\}\$} nebo \texttt{\textbackslash mbox\{\}} zabrání zalomení výrazu.

Pro definice a věty slouží prostředí definovaná příkazem \texttt{\textbackslash newtheorem} z balíku amsthm.
Tato prostředí obracejí význam \texttt{\textbackslash emph}:
uvnitř textu sázeného kurzívou se zvýrazňuje písmem v základním řezu.
Důkazy se někdy ukončují značkou \texttt{\textbackslash qed}.

\subsection{Pseudometrický prostor}
Pro zarovnání rovností a nerovnosti pod sebe použijte vhodné prostředí.

\begin{definition}
V pseudometrickém prostoru \emph{$\mathcal{M}=(M,\varrho)$ značí $M$ množinu bodů,
$\varrho: M\times M\rightarrow\mathbb{R}$ je zobrazení zvané \emph{pseudometrika}, které pro každé body $x,y,z\in M$ splňuje následující podmínky:}
\begin{eqnarray}
\label{eq1}\varrho(x,x)&=&0\\
\label{eq2}\varrho(x,y)&=&\varrho(y,x)\\
\label{eq3}\varrho(x,y)+\varrho(y,z)&\geq&\varrho(x,z)
\end{eqnarray}
\end{definition}

\subsection{Metrika}
Funkční hodnota pseudometriky $\varrho$ se nazývá \emph{vzdálenost}.
Vzdálenost každých dvou bodů je nezáporná.
\begin{sentence}\label{veta1}
\emph{Pro každé dva body $x,y\in M$ pseudometrického prostoru $(M,\varrho)$ platí $\varrho(x,y)\geq0$.}
\end{sentence}

Důkaz: Nechť $x,y\in M$ a označme $d=\varrho(x,y)$. Využitím \eqref{eq2} máme $2d=\varrho(x,y)+\varrho(y,x)$, z nerovnosti \eqref{eq3} vyplýva $2d\geq\varrho(x,x)$ a z nerovnosti \eqref{eq1} dostaneme $2d\geq\varrho(x,x)=0$. Odtud plyne $d\geq0$.\qed

Speciálním případem pseudometrických prostorů jsou prostory metrické, v nichž dva různé body mají vždy kladnou vzdálenost.

\begin{definition}\label{def2}
\emph{Nechť $\mathcal{M}=(M,\varrho)$ je pseudometrický prostor, v němž platí $\varrho(x,y)>0$, kdykoliv $x\neq y$. Potom $\mathcal{M}$ se nazývá metrický prostor a $\varrho$ je jeho metrika.}
\end{definition}

\section{Rovnice}
Velikost závorek a svislých čar je potřeba přizpůsobit jejich obsahu.
K tomu jsou určeny modifikátory \texttt{\textbackslash left} a \texttt{\textbackslash right}.

\begin{eqnarray}
\lim_{p\to 0} \left(\frac{1}{n} \sum\limits _{i=1}^n x_i^p\right)^\frac{1}{p}=\left(\prod\limits _{i=1}^n x_i\right)^\frac{1}{n}
\end{eqnarray}

Zde vidíme, jak se vysází proměnná určující limitu v běžném textu: $\lim_{m\to\infty}f(m)$.
Podobně je to i s dalšími symboly jako $\bigcup_{N\in\mathcal{M}}N$ či $\sum_{i=1}^m x_i^2$.
S vynucením méně úsporné sazby příkazem \texttt{\textbackslash limits} budou vzorce vysázeny v podobě $\lim\limits_{m\to\infty}f(m)$ a $\sum\limits_{i=1}^m x_i^2$.
Složitější matematické formule sázíme mimo plynulý text pomocí prostředí \texttt{displaymath}.

\begin{eqnarray}
\lim_{n\to\infty}\left(1+\frac{x}{n}\right)^n&=&\sum\limits_{n=0}^\infty \frac{x^n}{n!}\\
\sum\limits_{\emptyset\neq X\subseteq P} (-1)^{|X|-1}\left|\bigcap X\right|&=&\left|\bigcup P\right|\\
\label{eq7}-\int_{a}^{b}f(x)\:dx&=&\int_{b}^{a}f(y)\:dy
\end{eqnarray}
Nezapomeňte rovnice, na které se odkazujete, označit vhodným jménem pomocí \texttt{\textbackslash label}.

\section{Matice}
Pro sázení matic se používá prostředí \texttt{array} a závorky s výškou nastavenou pomocí
\texttt{\textbackslash left}, \texttt{\textbackslash right}.

$$
D=\left|\begin{array}{cccc}
a_{11}&a_{12}&\cdots&a_{1n}\\
a_{21}&a_{22}&\cdots&a_{2n}\\
\vdots&\vdots&\ddots&\vdots\\
a_{m1}&a_{m2}&\cdots&a_{mn}
\end{array}\right|=\left|\begin{array}{cc}
x&y\\
t&w 
\end{array}\right|=xw-yt$$

Prostředí \texttt{array} lze úspěšně využít i jinde,
například na pravé straně následující definiční rovnosti.
$$B_n=\left\{\begin{array}{l l}
1& \text{pro } n=0  \\
\sum\limits_{k=0}^{n-1}\binom{n}{k}B_k& \text{pro } n\geq 1
\end{array}\right.$$
Jestliže sázíme jen levou složenou závorku, pak za párovým \texttt{\textbackslash right}
místo závorky píšeme tečku.
\end{document}
